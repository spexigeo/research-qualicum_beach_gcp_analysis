\documentclass[11pt,a4paper]{article}
\usepackage[utf8]{inputenc}
\usepackage[T1]{fontenc}
\usepackage{graphicx}
\usepackage{booktabs}
\usepackage{geometry}
\usepackage{hyperref}
\usepackage{xcolor}
\usepackage{float}
\usepackage{caption}
\usepackage{subcaption}

\geometry{margin=2.5cm}
\hypersetup{
    colorlinks=true,
    linkcolor=blue,
    filecolor=blue,
    urlcolor=blue,
    citecolor=blue
}

\title{Orthomosaic Quality Comparison Report}
\author{Qualicum Beach GCP Analysis}
\date{2025-12-02T15:58:33.843044}

\begin{document}

\maketitle

\begin{abstract}
This comprehensive report compares orthomosaics generated with and without ground control points (GCPs)
against reference basemaps from ESRI World Imagery and OpenStreetMap. The analysis evaluates
absolute accuracy (RMSE, MAE), structural similarity, seamline artifacts, and 2D spatial
errors to assess the impact of GCPs on orthomosaic quality. The report provides detailed comparisons
for each basemap and concludes with recommendations on which method (with or without GCPs) provides
better results based on comprehensive analysis across all metrics.
\end{abstract}

\section{Methodology}

\subsection{Comparison Approach}
The orthomosaics are compared to reference basemaps using the following methodology:

\begin{enumerate}
\item \textbf{Reprojection}: The orthomosaic is reprojected to match the reference basemap's
      coordinate reference system (CRS) and spatial extent using bilinear resampling.
      
\item \textbf{Pixel-level Metrics}: 
\begin{itemize}
    \item \textbf{RMSE} (Root Mean Square Error): Measures overall pixel intensity differences
    \item \textbf{MAE} (Mean Absolute Error): Measures average absolute pixel differences
    \item \textbf{Structural Similarity}: Correlation-based measure of structural similarity
\end{itemize}

\item \textbf{Feature Matching}: Feature-based matching (SIFT, ORB, or phase correlation)
      is used to compute 2D spatial errors, providing X and Y offset measurements in pixels.
      This identifies systematic shifts or misalignments between the orthomosaic and reference.

\item \textbf{Seamline Detection}: Gradient-based analysis detects potential seamline artifacts
      by identifying high-gradient regions that may indicate stitching errors or discontinuities.
\end{enumerate}

\subsection{Reference Basemaps}
The orthomosaics are compared against multiple reference basemaps to ensure robust assessment:
\begin{itemize}
    \item \textbf{ESRI World Imagery}: High-resolution satellite imagery providing a detailed baseline
    \item \textbf{OpenStreetMap}: Community-sourced mapping data providing an alternative reference
\end{itemize}
Using multiple basemaps helps validate findings and account for potential biases in any single reference source.

\section{Quality Metrics}

\subsection{Overall Metrics Comparison - ESRI World Imagery}

\begin{table}[H]
\centering
\caption{Overall Quality Metrics Comparison (ESRI World Imagery)}
\begin{tabular}{lccc}
\toprule
Metric & Without GCPs & With GCPs & Improvement \\
\midrule
RMSE & 10.1513 & 10.1597 & -0.08\% \\
MAE & 152.9137 & 152.7436 & +0.11\% \\
Similarity & 0.5715 & 0.5719 & +0.06\% \\
Seamlines (\%) & 9.94 & 9.94 & -0.01\% \\
\bottomrule
\end{tabular}
\end{table}

\subsection{2D Spatial Error Metrics}

Feature matching provides spatial error measurements in pixels:

\begin{table}[H]
\centering
\caption{2D Spatial Error from Feature Matching}
\begin{tabular}{lcc}
\toprule
Metric & Without GCPs & With GCPs \\
\midrule
Mean X Offset (px) & 0.90 & -1.98 \\
Mean Y Offset (px) & -16.45 & -5.66 \\
2D RMSE (px) & 25.97 & 18.70 \\
Feature Matches & 15 & 14 \\
\bottomrule
\end{tabular}
\end{table}

\subsection{Overall Metrics Comparison - OpenStreetMap}

\begin{table}[H]
\centering
\caption{Overall Quality Metrics Comparison (OpenStreetMap)}
\begin{tabular}{lccc}
\toprule
Metric & Without GCPs & With GCPs & Improvement \\
\midrule
RMSE & 9.8338 & 9.8166 & +0.18\% \\
MAE & 103.6522 & 103.7207 & -0.07\% \\
Similarity & 0.4186 & 0.4185 & -0.03\% \\
Seamlines (\%) & 9.94 & 9.94 & -0.01\% \\
\bottomrule
\end{tabular}
\end{table}

\section{Visual Comparisons}

\subsection{Side-by-Side Comparison}

\begin{figure}[H]
\centering
\includegraphics[width=\textwidth]{visualizations/esri/comparison_side_by_side.png}
\caption{Comparison of orthomosaics with and without GCPs against the reference basemap.
The improvement map (bottom right) shows where GCPs reduce errors (green) or increase them (red).}
\label{fig:comparison}
\end{figure}

\subsection{Quality Metrics Summary}

\begin{figure}[H]
\centering
\includegraphics[width=0.8\textwidth]{visualizations/esri/metrics_summary.png}
\caption{Bar chart comparing quality metrics between orthomosaics with and without GCPs.}
\label{fig:metrics}
\end{figure}

\subsection{Seamline Detection}

\begin{figure}[H]
\centering
\begin{subfigure}{0.48\textwidth}
\centering
\includegraphics[width=\textwidth]{visualizations/esri/seamlines_no_gcps.png}
\caption{Without GCPs}
\end{subfigure}
\hfill
\begin{subfigure}{0.48\textwidth}
\centering
\includegraphics[width=\textwidth]{visualizations/esri/seamlines_with_gcps.png}
\caption{With GCPs}
\end{subfigure}
\caption{Seamline detection showing potential stitching artifacts. Red regions indicate
high-gradient areas that may represent seamlines or discontinuities.}
\label{fig:seamlines}
\end{figure}

\subsection{Error Maps - ESRI World Imagery}

\begin{figure}[H]
\centering
\begin{subfigure}{0.48\textwidth}
\centering
\includegraphics[width=\textwidth]{visualizations/esri/error_no_gcps.png}
\caption{Without GCPs}
\end{subfigure}
\hfill
\begin{subfigure}{0.48\textwidth}
\centering
\includegraphics[width=\textwidth]{visualizations/esri/error_with_gcps.png}
\caption{With GCPs}
\end{subfigure}
\caption{Error maps showing absolute differences between orthomosaics and ESRI World Imagery basemap.
Hotter colors indicate larger errors.}
\label{fig:errors}
\end{figure}

\subsection{Visual Comparisons - OpenStreetMap}

\subsubsection{Side-by-Side Comparison}

\begin{figure}[H]
\centering
\includegraphics[width=\textwidth]{visualizations/osm/comparison_side_by_side.png}
\caption{Comparison of orthomosaics with and without GCPs against the OpenStreetMap basemap.
The improvement map (bottom right) shows where GCPs reduce errors (green) or increase them (red).}
\label{fig:comparison_osm}
\end{figure}

\subsubsection{Quality Metrics Summary}

\begin{figure}[H]
\centering
\includegraphics[width=0.8\textwidth]{visualizations/osm/metrics_summary.png}
\caption{Bar chart comparing quality metrics between orthomosaics with and without GCPs (OpenStreetMap).}
\label{fig:metrics_osm}
\end{figure}

\subsubsection{Seamline Detection}

\begin{figure}[H]
\centering
\begin{subfigure}{0.48\textwidth}
\centering
\includegraphics[width=\textwidth]{visualizations/osm/seamlines_no_gcps.png}
\caption{Without GCPs}
\end{subfigure}
\hfill
\begin{subfigure}{0.48\textwidth}
\centering
\includegraphics[width=\textwidth]{visualizations/osm/seamlines_with_gcps.png}
\caption{With GCPs}
\end{subfigure}
\caption{Seamline detection showing potential stitching artifacts (OpenStreetMap comparison).
Red regions indicate high-gradient areas that may represent seamlines or discontinuities.}
\label{fig:seamlines_osm}
\end{figure}

\subsubsection{Error Maps}

\begin{figure}[H]
\centering
\begin{subfigure}{0.48\textwidth}
\centering
\includegraphics[width=\textwidth]{visualizations/osm/error_no_gcps.png}
\caption{Without GCPs}
\end{subfigure}
\hfill
\begin{subfigure}{0.48\textwidth}
\centering
\includegraphics[width=\textwidth]{visualizations/osm/error_with_gcps.png}
\caption{With GCPs}
\end{subfigure}
\caption{Error maps showing absolute differences between orthomosaics and OpenStreetMap basemap.
Hotter colors indicate larger errors.}
\label{fig:errors_osm}
\end{figure}

\section{Detailed Analysis}

\subsection{Root Mean Square Error (RMSE)}

The RMSE measures the overall pixel intensity differences between the orthomosaic and reference.
Without GCPs: 10.1513, With GCPs: 10.1597.
This represents a 0.08\% increase in error.

\subsection{Mean Absolute Error (MAE)}

The MAE measures the average absolute pixel differences.
Without GCPs: 152.9137, With GCPs: 152.7436.
This represents a 0.11\% improvement.

\subsection{Structural Similarity}

The similarity metric measures how well the orthomosaic structure matches the reference.
Without GCPs: 0.5715, With GCPs: 0.5719.
This represents a 0.06\% improvement in structural similarity.

\subsection{Seamline Artifacts}

Seamline artifacts are detected by analyzing gradient magnitudes.
Without GCPs: 9.94\% of pixels flagged, With GCPs: 9.94\%.
This represents a 0.01\% increase.

\subsection{2D Spatial Error}

Feature matching provides spatial error measurements indicating systematic shifts or misalignments.
Without GCPs: 25.97 pixels RMSE. With GCPs: 18.70 pixels RMSE.

\section{2D Shift Alignment Analysis}

After the initial comparison, 2D shifts were applied to align the orthomosaics with the ESRI basemap using feature matching. This section compares the results before and after alignment.

\subsection{Shift Parameters}

The feature matching algorithm computed the following pixel offsets to align each orthomosaic:

\begin{table}[H]
\centering
\caption{2D Shift Parameters Applied}
\begin{tabular}{lcc}
\toprule
Orthomosaic & X Shift (px) & Y Shift (px) \\
\midrule
Without GCPs & 5.00 & 24.85 \\
With GCPs & 7.99 & 21.38 \\
\bottomrule
\end{tabular}
\end{table}

\subsection{Comparison: Initial vs. Shifted Results}

The following table compares key metrics before and after applying 2D shift alignment:

\begin{table}[H]
\centering
\caption{Initial vs. Shifted Metrics Comparison (Without GCPs)}
\begin{tabular}{lccc}
\toprule
Metric & Initial & Shifted & Improvement \\
\midrule
RMSE & 10.1513 & 10.3752 & -2.21\% \\
MAE & 152.9137 & 155.5958 & -1.75\% \\
Similarity & 0.5715 & 0.3642 & -36.28\% \\
Seamlines (\%) & 9.94 & 11.63 & -1.69\% \\
\bottomrule
\end{tabular}
\end{table}

\begin{table}[H]
\centering
\caption{Initial vs. Shifted Metrics Comparison (With GCPs)}
\begin{tabular}{lccc}
\toprule
Metric & Initial & Shifted & Improvement \\
\midrule
RMSE & 10.1597 & 10.2660 & -1.05\% \\
MAE & 152.7436 & 152.2503 & +0.32\% \\
Similarity & 0.5719 & 0.4632 & -19.00\% \\
Seamlines (\%) & 9.94 & 11.38 & -1.44\% \\
\bottomrule
\end{tabular}
\end{table}

\subsection{Impact of 2D Shift Alignment}

The 2D shift alignment was applied to correct systematic georeferencing errors detected through feature matching. The results show:

For the orthomosaic without GCPs, alignment showed limited improvement.

For the orthomosaic with GCPs, alignment showed limited improvement.

This analysis demonstrates the potential for post-processing alignment corrections to improve orthomosaic accuracy, particularly when systematic georeferencing errors are present.

\section{Alignment Method Comparison}

Two post-processing alignment methods were evaluated: feature-matching alignment (using the ESRI basemap) and GCP-based alignment (using ground control points). This section compares their effectiveness.

\subsection{Comparison of Alignment Methods}

The following tables compare the results of both alignment methods:

\begin{table}[H]
\centering
\caption{Alignment Method Comparison (Orthomosaic Without GCPs)}
\begin{tabular}{lccc}
\toprule
Metric & Initial & Feature-Matched & GCP-Aligned \\
\midrule
RMSE & 10.1513 & 10.3752 & 10.1513 \\
MAE & 152.9137 & 155.5958 & 152.9137 \\
Similarity & 0.5715 & 0.3642 & 0.5715 \\
Seamlines (\%) & 9.94 & 11.63 & 9.94 \\
\bottomrule
\end{tabular}
\end{table}

\begin{table}[H]
\centering
\caption{Alignment Method Comparison (Orthomosaic With GCPs)}
\begin{tabular}{lccc}
\toprule
Metric & Initial & Feature-Matched & GCP-Aligned \\
\midrule
RMSE & 10.1597 & 10.2660 & 10.1597 \\
MAE & 152.7436 & 152.2503 & 152.7436 \\
Similarity & 0.5719 & 0.4632 & 0.5719 \\
Seamlines (\%) & 9.94 & 11.38 & 9.94 \\
\bottomrule
\end{tabular}
\end{table}

\subsection{Best Alignment Method}

For the orthomosaic without GCPs, the best alignment method is \textbf{Initial} with RMSE=10.1513.

For the orthomosaic with GCPs, the best alignment method is \textbf{Initial} with RMSE=10.1597.

\textbf{Recommendation: Initial orthomosaic alignment is sufficient.}

The initial orthomosaic alignment (without post-processing) provides the best results, suggesting that:
\begin{itemize}
    \item The original georeferencing is already accurate
    \item Post-processing alignment may introduce errors
    \item Additional alignment steps are not necessary
\end{itemize}

\section{Comprehensive Analysis and Recommendations}

\subsection{Summary of Results Across Basemaps}

The analysis compared orthomosaics with and without GCPs against multiple reference basemaps to ensure robust conclusions.

\subsubsection{Results Against ESRI World Imagery}

GCPs showed improvement in 2 out of 4 key metrics, with an average improvement of 0.02\%.

\subsubsection{Results Against OpenStreetMap}

Limited improvement observed with GCPs against OpenStreetMap. Only 1 out of 4 metrics showed improvement.

\subsubsection{Cross-Basemap Consistency}

Consistent degradations across both basemaps were observed for: SEAMLINE.

\subsection{Final Recommendation}

Based on comprehensive analysis across all metrics and reference basemaps:

\textbf{Recommendation: GCPs provide marginal benefits; consider based on project requirements.}

The analysis shows mixed results, with GCPs providing some improvements but not consistently across all metrics. Consider using GCPs if:
\begin{itemize}
    \item High absolute accuracy is critical for the project
    \item GCPs are readily available and accurately surveyed
    \item The additional processing time is acceptable
\end{itemize}

However, if processing speed is prioritized and the image alignment quality is already high, processing without GCPs may be sufficient.

\subsection{Additional Considerations}

When making the final decision, also consider:
\begin{itemize}
    \item \textbf{Project Requirements}: What level of accuracy is required for the intended application?
    \item \textbf{GCP Quality}: Are the available GCPs accurately surveyed and well-distributed?
    \item \textbf{Processing Time}: Can the project accommodate the additional processing time for GCP incorporation?
    \item \textbf{Cost-Benefit}: Does the improvement justify the cost of GCP collection and processing?
\end{itemize}

\end{document}
